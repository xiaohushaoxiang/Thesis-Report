%
% Abstract (does not appear in the Table of Contents)
\chapter*{Abstract}%

Pedestrian indoor localization is a problem  yet to have a solution that is as generally accepted as the Global Positioning System is for outdoor localization. An infrastructure independent option is to leverage the data from \ac{IMU} sensors to determine position. A large advantage of this approach is that these sensors are already available to many people since they are often integrated in the modern smartphone. Creating an indoor localization solution that works on this device could lead to wide adoption. This has been recognized by the research community, with methods already being tested on the device. An interesting combination worth considering is the combination of smartphone based indoor localization with the emerging market of wearable technology. This new technology has the potential to provide additional information that could improve indoor localization. One option is to detect activity that is not easily detectable from smartphones alone.

Within this thesis, a proof of concept is designed with the goal of finding a way in which activity recognition from wearable tech can help smartphone based indoor localization.
The implementation within this thesis relies only on the \ac{IMU} data of a smartwatch and smartphone, and map information of the indoor environment. It produces a position estimate of a pedestrian within the indoor environment. \par 

The system consists of two subsystems placed in series, where the output of the first subsystem, the \ac{SHS}, is the input of the second subsystem, the Particle Filter with  spatial context. The \ac{SHS} uses \ac{IMU} data from the smartphone to determine if a step is taken, what its length is, and in what direction it was taken. When multiple steps are taken a \ac{SHS} trajectory is generated as output of the subsystem.
The Particle Filter with spatial context uses the \ac{SHS} trajectory to generate position estimate in an indoor environment. It uses spatial constraints, such as walls, to limit the estimate. The \ac{IMU} data of a smartwatch is used to detection door interactions. These interactions in combination with known door locations, calibrates the position estimates to door locations. \par 

In order to evaluate performance, experiments was performed in an indoor environment where map information was available, with 6 trials being walked. Smartphone and smartwatch {IMU} data was recorded. Door interactions were also recorded manually.
Results from the experiments show that activity recognition can improve position estimates, however false positive in the recognition can have a detrimental effect on the position estimate.

