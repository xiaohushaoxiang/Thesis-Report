%
% Abstract (does not appear in the Table of Contents)
\chapter*{Abstract}%

Pedestrian indoor localization is a 

A proof of concept is designed with the goal of finding a way in which activity recognition can help indoor localization.
The implementation within this thesis relies only on the \ac{IMU} data of a smartwatch and smartphone and map information of the indoor environment. It produces a position estimate of the test subject within the indoor environment. \par 

The system consists of two subsystems placed in series, where the output of the first subsystem, the \ac{SHS}, is the input of the second subsystem, the Particle Filter with  spatial context. \par 

The \ac{SHS} uses \ac{IMU} data from the smartphone to determine if a step is taken, what its length is, and in what direction it was taken. When multiple steps are taken a \ac{SHS} trajectory is generated as output of the subsystem. \par 

The Particle Filter with spatial context uses the \ac{SHS} trajectory to generate position estimate in an indoor environment. It uses spatial constraints, such as walls, to limit the estimate. The \ac{IMU} data of a smartwatch is used to detection door interactions. These interactions in combination with known door locations, calibrates the position estimates to door locations. 

An experiment was performed in an indoor environment where map information was available, with 6 trials being walked. Smartphone and smartwatch {IMU} data was recorded. Door interactions were also recorded manually.

Results show that activity recognition can improve position estimates, however false positive in the recognition can have a detrimental effect on the position estimate.

