
\chapter{Introduction} \label{chap:intro}

The indoor localization of people has a large range of possible applications. It can be used for navigation in unknown environments, such as firefighters in an emergency situation or personal navigation through large public buildings \cite{Correa2017, Jackermeier2018}. Another application is the tracking of individuals, such as within elderly homes or workers in a factory setting \cite{Correa2017}. \par

Human indoor localization is an ongoing research field where there is yet a solution to be found that is widely accepted as the state of the art \cite{Davidson2017}. Unfortunately, the well-known solution for outdoor localization, the Global Positioning System (GPS) \cite{Jackermeier2018}, cannot be applied reliably in an indoor environment. GPS is a constellation of satellites circling the globe, emitting radio signals at specific frequencies. Once received, these signals can be used to triangulate a position. However, radio waves have difficulty in crossing solid structures reliably. Additionally, if a signal is received indoors, it may have been rebounded off of radio reflective surfaces, causing erroneous positioning estimates \cite{Jackermeier2018}. \par 

For any indoor localization method to be widely applicable, it is preferable that an indoor localization method provide accurate position estimations, is easily scalable, and has low system infrastructure cost \cite{Correa2017}.
In line with these applicability wishes, the smartphone has proven to be the perfect platform for implementation. It is an untethered device widely used in the developed world \cite{Correa2017}, containing many different sensors combined with computing and communication capabilities.
Creating a system that works well on this device will be crucial to launching indoor localization on a global scale \cite{Gu2019}. Focusing on this platform brings additional technical restrictions to what any eventual solution can use. This includes robustness in carrying modes and computational requirements.  For example, the continuous use of a camera is not beneficial for battery life, limiting the use of this method \cite{Yang2014, Solin2018a}. Due to the advantages that smartphones bring, they have been used often in indoor localization research \cite{Jackermeier2018,Correa2017,Yang2014, Qian2013}. \par 

Within the research field of indoor localization, the approaches are either infrastructure dependent or infrastructure independent, both using sensors on the person. Combined solutions have attempted to use the advantages of each to improve performance \cite{Gu2019, Correa2017}. The infrastructure \textit{dependent} methods imitate the GPS setup on a local scale. They often replace the GPS constellation with other signal-producing devices available within the indoor environment. Some of these systems, such as wireless fidelity(Wifi) and Bluetooth, are detectable through smartphone sensors, while wireless sensor networks (WSN), and ultra-wideband (UWB) require dedicated devices or attachments \cite{Wu2019,Jackermeier2018,Davidson2017}. For many of these solutions, detailed information on the position of the external signal-producing devices may be needed for estimation, including a signal strength map or device location \cite{Jackermeier2018,Shang2015}. This information may not be easily available. Furthermore, these solutions naturally only work in buildings with the installed equipment. This requires either that they are already present or some initial investment and setup is made. In some cases, this can not be expected. For example within subway systems, where legal or investment issues and potential vandalism can be a hindrance \cite{Torok2014}. \par
%
% Position estimation using these systems can be range based, leveraging signal time of arrival, signal angle or received signal strength. They can also be range free using proximity and fingerprinting to estimate position \cite{Correa2017, Tariq2017}

Infrastructure \textit{independent} positioning solutions do not need to receive signals from external devices. Since no external infrastructure is required, these methods can have low infrastructure cost. Many such methods make use of cameras or inertial sensors placed on the target (sensors often found in smartphones). Camera-based systems use unique visual features tracked in subsequent frames to determine displacement \cite{Gu2019}. With such an approach, privacy concerns, computing complexity, and device placement need to be considered \cite{Gu2019}. Inertial sensor-based methods use \ac{IMU} sensors that measure linear acceleration and angular velocity. They are \ac{MEMS},  consisting of triaxial orthogonal accelerometers and gyroscopes, and can include a triaxial magnetometer \cite{Yang2014}. These sensors are relatively small, cheap, and have low power consumption \cite{Olsson2016}. This makes them portable, easily accessible, and able to work long periods of use, and therefore potential long-term localization. These sensors suffer from noise and biases, eventually leading to estimate drift. Different methods exist that try and reduce this drift. The methods that use these sensor for human location estimation are known as \ac{PDR}. \par

The combination of a smartphone with a wearable device is an important development in indoor localization technology. While the smartphone market has matured, the market for wearable devices has been growing steadily \cite{jung2016consumer}, with the smartwatch being the most prominent. These devices offer similar sensors and capabilities as smartphones, with the same advantages and disadvantages. One form of information that they can provide is the detection of interactions with the indoor environment, also known as activity recognition. These interactions may not be easily detectable through smartphone sensors alone \cite{Shoaib2015}. This can help with localization since some indoor interactions can be position-dependent. Examples include interacting with doors, stairs, and furniture. By knowing the location of these structures within the indoor environment, a position measurement is available if a relevant interaction is detected. This information can be combined with smartphone based systems to potentially improve indoor localization.

There is previous research that uses different forms of activity recognition to aid in localization. In some cases this can be a simple approach, such as applying a threshold on orientation change to determine if a corner has been turned or not. Once detected, corner definitions from map information is used to improve localization \cite{Gu2019,Jackermeier2018}. \par 
\citet{Hardegger2012} created ActionSLAM, which uses a foot mounted sensor to estimate displacement in combination with location-related actions to compensate any drift buildup. It does not require map information as it used \ac{SLAM}, to produce a map itteratively. In the case of ActionSLAM, the activity recognition was performed manually by annotating from video recordings. The paper indicates a tracking error between a ground truth and the calculated path to be around 1.3 meters over 300 meters of movement. This research is augmented in \citet{Hardegger2016} where they have recognized the growing trend of wearable technology. They derived a method that used a \ac{PDR} system in combination with activity recognition from inertial sensors through template matching, Particle Filters, and \ac{SLAM} to improve both activity recognition and localization. Examples of detected actions are opening a window, picking up a phone and opening a drawer. \par
A different approach was used by \citet{Grzonka2010}, who used a body suit composed of multiple \ac{IMU} sensors to detect the opening and closing of doors as landmarks, also using a \ac{SLAM} approach for map generation. \citet{Altun2012} use a similar body suit in combination with map information and activity recognition of activities such as standing up from a chair to improve localization. \citet{Torok2014} have developed DREAR, which is a Hidden Markov Model to detect landmarks using a combination of map information and smartphone sensors, both inertial sensor and magnetometer, to detect when a person is sitting, walking, and taking an escalator.

\section{Research Question}

So far, the PDR methods that uses activity recognition to improve indoor localization have either used dedicated sensors or have been only smartphone based. There is yet to be a \ac{PDR} approach that is smartphone based and uses wearable technology for activity recognition. This leads to the research question of this thesis, which focuses on smartwatches, the most prominent wearable tech at the moment. The research question is:

\textbf{How can smartwatch based activity recognition improve a smartphone based \ac{PDR} system with indoor localization?}

\section{Research Scope}

The scope of this thesis will be on the \textit{combination} of activity recognition from wearable technology and smartphone based indoor localization. A simple form of activity recognition will be used where door interaction is detected. The performance of this activity recognition scheme will not be the focus of research. It will be a vessel for showing the advantages and disadvantages of using this form of position measurement together with \ac{SHS}, and the different aspects that will need to be considered for future work.

\section{Thesis Contributions}

In this thesis I am further exploring the combined technology of smartphones and wearable device by investigating how \ac{IMU} sensors in a smartwatch can be used to improve smartphone IMU-based indoor localization. I will present the different \ac{PDR} methods available. I will be focusing on a specific \ac{PDR} method, namely \ac{SHS} since it seems the most applicable for smartphone implementation \cite{Kang2015}. This will be combined with different drift reduction techniques, including a map based Particle Filter augmented with landmark detection through smartwatch activity recognition. A proof of concept will be designed and tested in a realistic environment. Results will be summarized and analyzed, indicating the performance and limitations of the approach.

\section{Thesis Outline}
%A solution with the potential to meeting the above requirements is Pedestrian Dead Reckoning (PDR), a technique where the displacement from an initial location is calculated for a pedestrian. This technique often uses inertial sensors, with different possible placements on the body \cite{Gu2019}. These sensors are \ac{MEMS}, and consist of triaxial orthogonal accelerometers and gyroscopes, and can include a triaxial magnetometer \cite{Yang2014}. These sensors are small, lightweight and cheap, and are found in almost all smartphone brands.
%While able to generate a position estimate for the pedestrian, position estimation drift is inevitable



%Dead reckoning (DR) presents an interesting, incremen­tal positioning modality for pedestrians, complementary to absolute positioning of modern mobile phones (GPS, WiFi, Bluetooth and others). Many location aware applications can profit from greater accuracy and indoor availability, that can be achieved by combining DR with absolute positioning modalities. While dead reckoning is limited for long-term use due to error accumulation, it can achieve high short- to mid-term accuracy, and can be used to gap non-availability phases ^citation from Dead Reckoning from the Pocket - An Experimental StudyUlrich Steinhoff* and Bernt Schiele*t
 
%Although global positioning system (GPS)–based outdoor localization is very common, indoor localization remains a challenge because GPS signals are unavailable indoors. Many indoor localization methods are based on wireless radio systems, such as Wi-Fi [2], radio-frequency identity [3], ultra-wideband [4], and Bluetooth [5]. These localization methods can be categorized into two types [6]: triangulation and fingerprinting. The former relies on installed expensive hardware, making it neither scalable nor universal. The latter requires pretraining, which is time-consuming. citation from A Robust Step Detection and Stride Length Estimation for Pedestrian Dead Reckoning Using a Smartphone 


%Repurposing smartphones for ubiquitous sensing is challenging due to their battery constraints and multi-purpose nature. We observed that each embedded sensor has a similar power draw if used individually, but using combinations of sensors costs significantly more (see Figure 1). Thus, we use solely the accelerometer (applicable even to early smartphones). Smartphones are not dedicated sensing devices and they suffer from dropped samples and significant jitter in signal timestamps. Moreover, their nonrigid attachment (they may be carried in front pockets; back pockets; shirt pockets; hands; bags and more) and freedom of motion violate many of the assumptions made in previous step detection work. citation from Brajdic and harle 2013 
