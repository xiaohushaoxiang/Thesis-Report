
\chapter{Introduction} \label{chap:intro}

%Dead reckoning (DR) presents an interesting, incremen­tal positioning modality for pedestrians, complementary to absolute positioning of modern mobile phones (GPS, WiFi, Bluetooth and others). Many location aware applications can profit from greater accuracy and indoor availability, that can be achieved by combining DR with absolute positioning modalities. While dead reckoning is limited for long-term use due to error accumulation, it can achieve high short- to mid-term accuracy, and can be used to gap non-availability phases ^citation from Dead Reckoning from the Pocket - An Experimental StudyUlrich Steinhoff* and Bernt Schiele*t
 
%Although global positioning system (GPS)–based outdoor localization is very common, indoor localization remains a challenge because GPS signals are unavailable indoors. Many indoor localization methods are based on wireless radio systems, such as Wi-Fi [2], radio-frequency identity [3], ultra-wideband [4], and Bluetooth [5]. These localization methods can be categorized into two types [6]: triangulation and fingerprinting. The former relies on installed expensive hardware, making it neither scalable nor universal. The latter requires pretraining, which is time-consuming. citation from A Robust Step Detection and Stride Length Estimation for Pedestrian Dead Reckoning Using a Smartphone 


%Repurposing smartphones for ubiquitous sensing is challenging due to their battery constraints and multi-purpose nature. We observed that each embedded sensor has a similar power draw if used individually, but using combinations of sensors costs significantly more (see Figure 1). Thus, we use solely the accelerometer (applicable even to early smartphones). Smartphones are not dedicated sensing devices and they suffer from dropped samples and significant jitter in signal timestamps. Moreover, their nonrigid attachment (they may be carried in front pockets; back pockets; shirt pockets; hands; bags and more) and freedom of motion violate many of the assumptions made in previous step detection work. citation from Brajdic and harle 2013 


\section{Research Questions}

Can indoor localization be achieved with realistic phone placement?
Can activity recognition improve the above derived method 

\section{Thesis Contributions}