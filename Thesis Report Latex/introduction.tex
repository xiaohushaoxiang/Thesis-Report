
\chapter{Introduction} \label{chap:intro}

Indoor localization is an ongoing research field where there is yet a solution to be found that is widely accepted as the state of the art \cite{Davidson2017}. For outdoor localization, the de facto solution is the Global Positioning System (GPS) \cite{Jackermeier2018}. It is a constellation of satellites circling the globe, emitting radio signals at specific frequencies. Once received, these signals can be used to triangulate a position. \par 

The GPS solution can not be applied reliably in an indoor environment. This is because of the difficulty of radio waves in crossing solid structures reliably. Additionally, if a signal is received indoors, it may have been rebounded off of radio reflective surfaces, causing erroneous positioning estimates \cite{Jackermeier2018}. \par 

As with outdoor localization, indoor localization has a large range of possible applications. It can be used for navigation in unknown environments, such as firefighters in an emergency situation or personal navigation through large public buildings \cite{Correa2017, Jackermeier2018}. Another application is the tracking of individuals, such as within elderly homes or workers in a factory setting \cite{Correa2017}. For wide applicability it is preferable that an indoor localization method provide accurate position estimations, is easily scalable, and has low system infrastructure cost \cite{Correa2017}.\par

In line with these applicability wishes, the smartphone has proven to be the perfect platform for implementation.
It is an untethered device widely used in the developed world \cite{Correa2017}, containing many different sensors combined with computing and communication capabilities.
Creating a system that works well on this device will be crucial to launching indoor localization on a global scale \cite{Gu2019}. Focusing on this platform brings additional restrictions to what any eventual solution can use. This includes robustness in carrying modes and computational requirements.  For example, the continuous use of a camera is not beneficial for battery life, limiting the period in which this method can be used \cite{Yang2014, Solin2018a}. Due to the advantages that smartphones bring, they have been used often in indoor localization research \cite{Jackermeier2018,Correa2017,Yang2014, Qian2013}. \par 

Within the research field of indoor localization, the approaches are either infrastructure dependent or not, both using sensors on the target. Combined solutions have attempted to use the advantages of each to improve performance \cite{Gu2019, Correa2017}.\par

The infrastructure dependent methods imitate the GPS setup on a local scale. They often replace the GPS constellation with other signal-producing devices available within the indoor environment. Some of these systems, such as wireless fidelity(Wifi) and Bluetooth, are detectable through smartphone sensors, while wireless sensor networks (WSN), and ultra-wideband (UWB) require dedicated devices or attachments \cite{Wu2019,Jackermeier2018,Davidson2017}. For many of these solutions, detailed information on the position of the external devices may be needed for estimation, including a signal strength map or device location \cite{Jackermeier2018}. This information may not be easily available. Furthermore, these solutions naturally only work in buildings with the installed equipment. This requires either that they are already present or some initial investment and setup. In some cases, this can not be expected. For example within subway systems, where legal or investment issues and potential vandalism can be a hindrance \cite{Torok2014}. \par
%
% Position estimation using these systems can be range based, leveraging signal time of arrival, signal angle or received signal strength. They can also be range free using proximity and fingerprinting to estimate position \cite{Correa2017, Tariq2017}

Infrastructure independent positioning solutions do not need to receive signals from external devices. Since no external infrastructure is required, these methods can have low infrastructure cost. Many such methods make use of cameras or inertial sensors placed on the target; sensors often found in smartphones. Camera-based systems use unique visual features tracked in subsequent frames to determine displacement \cite{Gu2019}. With such an approach, privacy concerns, computing complexity, and device placement need to be considered. Inertial sensor-based methods use sensors that measure linear acceleration and angular velocity. They are \ac{MEMS},  consisting of triaxial orthogonal accelerometers and gyroscopes, and can include a triaxial magnetometer \cite{Yang2014}. These sensors are relatively small, cheap, and have low power consumption \cite{Olsson2016}. Low power consumption allows for long periods of use, and therefore long-term localization. The methods that use these sensor for location estimation are known as \ac{PDR}. \par

While the smartphone market has matured, the market for wearable devices has been growing steadily \cite{jung2016consumer}, with the smartwatch being the most prominent. These devices offer similar sensors and capabilities as smartphones, with the same advantages and disadvantages. One form of information that they can provide is the detection of interactions with the indoor environment, also known as activity recognition. These interactions may not be easily detectable through smartphone sensors alone \cite{Shoaib2015}. This can help with localization since some indoor interactions can be position-dependent. Examples include interacting with doors, stairs, and furniture. By knowing the location of these structures within the indoor environment, a position measurement is available if a relevant interaction is detected. This information can be combined with the data from smartphone sensors to potentially improve indoor localization.
This leads to the overall goal of this thesis:

\textbf{How can IMU sensors in a smartwatch be used with activity recognition to improve smartphone-based pedestrian dead reckoning?}

%A solution with the potential to meeting the above requirements is Pedestrian Dead Reckoning (PDR), a technique where the displacement from an initial location is calculated for a pedestrian. This technique often uses inertial sensors, with different possible placements on the body \cite{Gu2019}. These sensors are \ac{MEMS}, and consist of triaxial orthogonal accelerometers and gyroscopes, and can include a triaxial magnetometer \cite{Yang2014}. These sensors are small, lightweight and cheap, and are found in almost all smartphone brands.
%While able to generate a position estimate for the pedestrian, position estimation drift is inevitable



%Dead reckoning (DR) presents an interesting, incremen­tal positioning modality for pedestrians, complementary to absolute positioning of modern mobile phones (GPS, WiFi, Bluetooth and others). Many location aware applications can profit from greater accuracy and indoor availability, that can be achieved by combining DR with absolute positioning modalities. While dead reckoning is limited for long-term use due to error accumulation, it can achieve high short- to mid-term accuracy, and can be used to gap non-availability phases ^citation from Dead Reckoning from the Pocket - An Experimental StudyUlrich Steinhoff* and Bernt Schiele*t
 
%Although global positioning system (GPS)–based outdoor localization is very common, indoor localization remains a challenge because GPS signals are unavailable indoors. Many indoor localization methods are based on wireless radio systems, such as Wi-Fi [2], radio-frequency identity [3], ultra-wideband [4], and Bluetooth [5]. These localization methods can be categorized into two types [6]: triangulation and fingerprinting. The former relies on installed expensive hardware, making it neither scalable nor universal. The latter requires pretraining, which is time-consuming. citation from A Robust Step Detection and Stride Length Estimation for Pedestrian Dead Reckoning Using a Smartphone 


%Repurposing smartphones for ubiquitous sensing is challenging due to their battery constraints and multi-purpose nature. We observed that each embedded sensor has a similar power draw if used individually, but using combinations of sensors costs significantly more (see Figure 1). Thus, we use solely the accelerometer (applicable even to early smartphones). Smartphones are not dedicated sensing devices and they suffer from dropped samples and significant jitter in signal timestamps. Moreover, their nonrigid attachment (they may be carried in front pockets; back pockets; shirt pockets; hands; bags and more) and freedom of motion violate many of the assumptions made in previous step detection work. citation from Brajdic and harle 2013 

\newpage
\section{Research Questions}
\textcolor{red}{ATM I am not very sure on these research questions. I would like to discuss with you better options!}

In order to tackle the subject, two main research questions can be defined, with further subquestions and explanation below them.

\begin{itemize}
	\item \textbf{How can indoor localization be achieved with realistic smartphone placement, while taking computation capabilities into account?}
	\begin{itemize}
		\item What indoor localization methods consider realistic smartphone carrying modes? 
	\end{itemize}
	\item\textbf{How can wearable devices improve indoor localization?}
	\begin{itemize}
		\item What information can be derived from wearable devices?
	\end{itemize}
\end{itemize}

\section{Thesis Contributions}
This thesis will explore the different possibilities for realistic smartphone-based indoor localization and how wearable devices can aid in improving a position estimate. It will explore a proof of concept with the intention of showing how data from wearable technology can improve localization. This thesis will test the designed system's performance in experiments, highlight the different advantages, disadvantages, and limitations. Future steps are outlined on how to improve the derived method for future research.

\section{Thesis Outline}
\textcolor{cyan}{Thesis outline will be completed once thesis structure has solidified}