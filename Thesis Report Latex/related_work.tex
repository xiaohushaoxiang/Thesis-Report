% With INS the position error grows cubically with time, due to the signal to noise ratio inherent to MEMS IMU sensors \cite{Harle2013}. To compensate for this error, it is preferable that the IMU sensor is mounted on the foot, as will be explained in \cref{sec:INS}. This placement requirement may limit usability. \\

%One advantage of the SHS implementation is that the position error is proportional to the number of steps taken, but without the preference of using foot-based sensors, allowing for other sensor placements \cite{Diez2018b}.

% This section will outline the different PDR methods and focus on an SHS approach to indoor localization.

\chapter{Step and Heading System}

\subsubsection{Walk and Step Detection}
\label{sec:rw - step detection}
Walk detection is determining from sensor data whether a walking activity is being performed. Step detection is deciding when during the walking activity a step is taken. \\
Techniques such as feature classification, frequency domain analysis, and time-domain thresholding can be used to detect the two activities \cite{Yang2014}. An overview of techniques  is  shown in \cref{fig:step_detection_options} and summarized in \cref{tab:step_detection_comparison}. %The reader is referred to \cite{Brajdic2013} for a detailed explanation of each solution.

Naturally, each form of analysis for walk and step detection has its advantages and disadvantages.\\
Time-domain analysis often uses the acceleration norm, making it robust to sensor orientation \cite{Davidson2017}. Its simplest form, thresholding, is trivially simple to implement. The problem with thresholding is that it is difficult to determine the optimal value, as it can vary between users, surfaces, and even shoes \cite{Brajdic2013}. \\
Similar to time-based methods, the periodicity in the norm of acceleration during locomotion is frequently invariant to sensor placement. This therefore also makes frequency analysis robust to sensor orientation. Frequency methods have the disadvantage that they are only able to detect periodic motion. 
For both frequency and time domain-based analysis, it is possible that certain motions, aside from the desired motion, can cross the threshold and generate false positives. In addition, frequency analysis and template matching, a more advanced time based method, will have a certain computational overhead to consider \cite{Davidson2017, Harle2013}. \\
Feature classification methods use features in accelerometer traces to classify steps, using form of machine learning. These method have the advantage that they can determine relations from labeled data automatically, removing a need for humans to define them. This is also a disadvantage as it requires labeled data, the gathering of which can be a labor-intensive process \cite{Bulling2014}. Furthermore, the relationships they determine will depend on the data gathered. This may lead to person dependent performance [\qn]. In addition, certain classification methods may require significant processing power, limiting the platforms on which it can perform [\qn].


\begin{figure}
	\centering
	\includegraphics[trim=0 15 0 20, clip,width=0.7\linewidth]{images/step_detection_options}
	\caption{Overview of walk and step detection methods}
	\label{fig:step_detection_options}
\end{figure}

	\begin{table}
	\centering
	\footnotesize
	\begin{tabularx}{\linewidth}{@{} P{17mm} L}
		\toprule
		\thead[lb]{Technique}	&   \thead[c]{Explanation}	\\
		\midrule			
		Threshold & Acceleration magnitude is monitored for the passing of certain values or sign changes, in the case of Zero Crossing method \cite{Davidson2017,Harle2013}. Typically determines when the foot is on the floor \cite{Harle2013}, but has also been applied to pitch angle of the upper leg, where the sensor has been placed \cite{Diaz2014a}. In some instances, the thresholds are altered adaptively, for example through different motion mode detection or information gathered from the previous step \cite{Wu2019}.\\ \cline{1-2}
		(Windowed) Peak Detection & Recognizes local maximum or minimum on acceleration magnitude caused by foot impact on the floor, often within a sliding window \cite{Susi2013}. Generally combined with thresholding, which is one of the simplest combination used with \ac{SHS} \cite{Davidson2017}. \\ \hline
		Auto-correlation (Template Matching) &  Finds correlation of a signal with a time-shifted copy of itself. It leverages the strong cyclic nature of bipedal locomotion \cite{Harle2013}. Since the lag for the highest correlation is not known beforehand, an interval is often swept and correlation values compared. \\ \hline
		Dynamic Time Warping & Similar to autocorrelation, it measures the similarity between two waveforms from accelerometer data \cite{Davidson2017}, These waveforms can both come from the data or one could be a stride template generated offline. A non linear mapping between the two methods is made, resulting in a DTW distance. A smaller distance indicates higher similarity \cite{Davidson2017}.  \\ \hline
		Short Term Fourier Transform & Transforms acceleration signal of successive windows of data into the frequency domain. For spectral analysis, a subset containing a stride (two steps) is required to determine the frequency \cite{Harle2013}.  \\ \hline
		Wavelet Decomposition & Acceleration magnitude is split into low and high-frequency components, from which the dominant frequency is assumed to be the walking frequency. Iterating this process on the resulting low-frequency signal approximation, a smoother shaped dominant low-frequency signal is generated. The frequency of this signal corresponds to walking cadence \cite{Davidson2017}. \\ \hline	
		Hidden Markov Model & Uses gait cycles segmented into different states of a state machine, to determine if a step has been taken \cite{Ren2016a}. Thresholds can be used to induce a state change. If a full state machine cycle is achieved, a step is detected.\\ \hline
		K Nearest Neighbours & Uses labeled data containing features from successive time windows and compares a new time window with its features. It finds the labeled data whose features are most similar. This new set then recieves its label. \\
		\bottomrule
	\end{tabularx}
	\caption{Overview of different step detection methods}
	\label{tab:step_detection_comparison}
\end{table}

In \cite{Brajdic2013}, a variety of algorithms for both walk detection and step detection on unconstrained smartphones are tested and compared. The paper reviews techniques with different levels of complexity from a variety of different papers. In order to compare the different techniques, \citet{Brajdic2013} collected a large dataset from 27 test subjects leading to 130 recordings was made. Each subject held the smartphone in six different carrying modes: in hand in front of user (idle and with device interaction), in a front pocket, in a back pocket and in a handbag or backpack. A ground truth was generated by video recording each session and manually counting the amount of steps taken. Using this dataset a comparison is made between the nine algorithms.
\newline
The paper concludes that for the generated dataset, the best step counting results were obtained using the windowed peak detection, hidden Markov model, and continuous wavelet transform. Each had a median error of about 1.3\%.\\
Considering the relative simplicity of the technique, the authors recommend that windowed peak detection as the most efficient algorithm for step detection. The best walk detection algorithms were thresholds on either standard deviation or signal energy,  short-term Fourier transform, and normalized autocorrelation.\\
\citet{Salvi2018} build upon the conclusions and recommendations of \citet{Brajdic2013}, with the aim of further optimizing the windowed peak detection algorithm and its parameters. The algorithm is based on the approach of \citet{Palshikar2009} and consists of 5 stages run in series. These are pre-processing, filtering, scoring, detection and post processing. An exhaustive grid search across the parameter space was performed to find the optimal set for step detection. Using these parameters, an average accuracy of $95\% \pm 4.5\%$ was reached for the different carrying modes. 

%These results make this approach 


\subsubsection{Step Length Estimation}
In order to generate a displacement vector from step detection, the step length must be estimated. \citet{Collins2013a} found that increasing step speed leads to larger step lengths, while step speed can have slow and spontaneous fluctuations depending on the motion mode. In addition, step size depends on the physical characteristics of the user and on their walk strategy, which can be different per individual \cite{Diez2018}. These discrepancies indicate that using a simple average step length for every pedestrian could result in quick accumulation of error. 

\citet{Diez2018} categorizes step length estimation methods into integration based and model-based methods. \\
Theoretically, the double integration of the IMU acceleration signal is the best approach to step size estimation, using the INS approaches outlined in chapter \secref{sec:INS}. This would give a direct measurement of displacement. It does not require any modeling, assumptions, or person-specific calibration \cite{Diez2018}. However, since it would be an INS approach, it suffers from the same drift problems and would require the same solutions. Therefore this approach benefits from having the sensor to be located on the foot. \\
Analytical models can be made of human mobility based on geometrical relationships of body composition, angles, and displacement of body parts. One of the largest disadvantages of a model-based approach is that human proportions are not uniform, requiring approximations and/or some form of calibration for the model to be accurate.
An overview of different step length estimation methods can be found in \cref{tab:step_length_methods}.

\begin{table}[H]
	\centering
	\textcolor{cyan}{Table with different step length methods}
	\caption{Different Step Length Methods}
	\label{tab:step_length_methods}
\end{table}

\citet{Vezocnik2019} compared different existing step length estimation algorithms. The review focuses on the methods applicable to smartphone use. This means methods that do not require training and do not require the sensor to be placed on the foot. This, therefore, excludes machine learning and INS systems. The models used are either based on an inverted pendulum model or relate predictors to step length. Examples of step length predictors are step frequency and acceleration range within a step. The robustness of the methods was tested by having the smartphone in different carrying modes. This includes front 

\begin{equation}
	\label{eq:Tian2016_sle}
	\text{step size} = K \cdot h \cdot \sqrt{F}.
\end{equation}

Here $K$ is a tunable parameter, $h$ is the height of the user and $F$ is the step frequency. This method reported an average error of  4.59 \% for personalized variables and 6.96 \% for global ones. For personally tuned variables the method of \cite{Weinberg2002} was best. This is an inverted pendulum model in which the human center of mass is used, located approximately at the pelvis. The center of mass rotates as an inverted pendulum when taking a step. This is followed by a forward horizontal displacement when both feet are on the ground \cite{Diez2018}. The model is defined as 

\begin{equation}
\text{step size} =K \sqrt[4]{A_{\max }-A_{\min }}.
\label{eq:weinberg_stepsize}
\end{equation}

Here $A_{\max}$ is the largest measured acceleration measured within a step interval, while $A_{\min}$ is the smallest. $K$ is a calibration variable  \cite{Weinberg2002,Diez2018}. The model had an average error of  10.64 \% for global variables and  3.60 \% for personalized \cite{Vezocnik2019}.



\subsubsection{Step Heading Estimation}
Step heading determines the direction of a detected step. It requires the orientation of the sensor in the navigation frame and determining in what direction the sensor is moving in the sensor frame. This provides an estimate in which direction the sensor is moving in the navigation frame.  Step heading estimation is currently the component within SHS whose performance is the most limiting for positioning purposes \cite{Diez2018b, Qian2013,Combettes2017}.\\
Even though the phone orientation may be known accurately, the direction in which the user is moving is not instantly clear. \par
There are two approaches to determining the heading. The first is knowing beforehand what the orientation of the phone is with respect to heading \cite{Tian2016}.  This would constrain the carrying mode that can be used.  With the use of motion classification, this method could be expanded, where different carrying modes can be sensed,which changes the heading accordingly. Heading per carrying mode would need to be derived beforehand, through the training of the necessary model. 

\textcolor{cyan}{this section is not done yet, and so will need completing } \\ \newline
\textcolor{red}{I do not have a comparison of different orientation estimation methods. How do I explain the use of EKF compared to the different possibilities? Should i just add this comparison section?}

\subsubsection{Activity Recognition}





% The Beauregard implementation introduced backtracking, whereby the filter kept a limited history of each particle’s ancestors to allow deletion of an entire trajectory when a particle was killed due to a wall constraint. This is a variant of backward belief propagation also used by Rai et al. [23] and is useful to improve position estimates made in the past when live positioning is not a requirement.
